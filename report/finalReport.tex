\documentclass[10pt,twocolumn,letterpaper]{article}

\usepackage{cvpr}
\usepackage{times}
\usepackage{epsfig}
\usepackage{graphicx}
\usepackage{amsmath}
\usepackage{amssymb}

% Include other packages here, before hyperref.

% If you comment hyperref and then uncomment it, you should delete
% egpaper.aux before re-running latex.  (Or just hit 'q' on the first latex
% run, let it finish, and you should be clear).
\usepackage[breaklinks=true,bookmarks=false]{hyperref}

\cvprfinalcopy % *** Uncomment this line for the final submission

\def\cvprPaperID{****} % *** Enter the CVPR Paper ID here
\def\httilde{\mbox{\tt\raisebox{-.5ex}{\symbol{126}}}}

% Pages are numbered in submission mode, and unnumbered in camera-ready
%\ifcvprfinal\pagestyle{empty}\fi
\setcounter{page}{4321}
\begin{document}

%%%%%%%%% TITLE
\title{Hierarchical Multiclass Object Classificaiton }

\author{First Author\\
Institution1\\
Institution1 address\\
{\tt\small firstauthor@i1.org}
% For a paper whose authors are all at the same institution,
% omit the following lines up until the closing ``}''.
% Additional authors and addresses can be added with ``\and'',
% just like the second author.
% To save space, use either the email address or home page, not both
\and
Second Author\\
Institution2\\
First line of institution2 address\\
{\tt\small secondauthor@i2.org}
}

\maketitle
%\thispagestyle{empty}

%%%%%%%%% ABSTRACT
\begin{abstract}

Human can use similarity between objects in order to recognize rare objects. They also make many abstract concepts when they see some objects very often. Interestingly, a large part of brain is associated with common classes like faces rather than rare objects like Ostrich.
In our work we want to propose a model that has four mentioned characteristics. 1. Use more resources for categories that have many examples and less resources for categories that have few examples. 2. Has the ability recognize objects that it has never seen before, but can be made by combining previously seen objects. 3. Objects with few examples can borrow statistical strength from similar objects with many examples. 4. Models that have lot of sharing between model parameters of different classes should be favored.

\end{abstract}

%%%%%%%%% BODY TEXT
\section{Introduction}

\section{Previous work}


\section{Hierarchical Classification Model}


{\small
\bibliographystyle{ieee}
\bibliography{egbib}
}

\end{document}
